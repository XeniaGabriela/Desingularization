\documentclass{article}
\usepackage{amsfonts}
\usepackage{amsmath}
\usepackage{amssymb}
\usepackage{color}
\usepackage{graphicx}
\usepackage{anyfontsize}
%\usepackage{epsfig}
%\usepackage[all]{xy}
\newcounter{excnt}
\newtheorem{thm}{Theorem}
\newtheorem{df}[thm]{Definition}
\newtheorem{prop}[thm]{Proposition}
%\newtheorem{rem}[thm]{Remark}
\newtheorem{cor}[thm]{Corollary}
\newtheorem{lem}[thm]{Lemma}
\newenvironment{proof}%
     {\noindent {\bf Proof:}}{\nopagebreak\begin{flushright}{\bf q.e.d.}\end{flushright}}
\newenvironment{example}%
     {\stepcounter{excnt}
      \noindent {\bf Example \arabic{excnt}} \phantom{i}}{\vspace{0.4cm}}
\newenvironment{algo}[1]{\begin{quote}{\bf Algorithm #1}}{\end{quote}}

\title{Lipmans Desingularization Algorithm}
\author{
	apl. Prof. Dr. Anne Fr\"uhbis--Kr\"uger (Institut f. Alg. Geometrie, \\
        Leibniz Universt\"at Hannover, Germany)\\
 	Prof. Dr. Florian Heß (Institut für Mathematik, \\
        Universt\"at Oldenburg, Germany)\\
	Xenia Bogomolec (Institut f. Alg. Geometrie, \\
        Leibniz Universt\"at Hannover, Germany)
}

% Florian Hess nicht vergessen


\begin{document}

\maketitle

\abstract{}

\section{Introduction}

\section{Lipmans approach} 
\label{sec2}

The strategy of separating a singular point of a curve into several points and 
even resolve the singularity by blowing it up successively leads to the 
desingularization of the curve. It is the statement of the following proposition.\\


\begin{prop}[Liu06, Prop. 8.1.26]
Let $X \rightarrow$ $spec(k)$ be an integral projective curve over a field $k$.
Then the sequence
$$ \cdots X_{n}\rightarrow X_{n-1}\rightarrow\cdots\rightarrow X_1\rightarrow X_0=X$$
is finite whereat $X_{i}\rightarrow X_{i-1}$ is the blow up of $X_{i-1}$ along the 
singular locus $Sing(X_{i-1})$ which is a closed subset of $X_{i-1}$.
\end{prop}

\noindent
If we have a surface over spec$(k)$ we can modify the process of successfully 
blowing up points by normalizing the surface first. Normal rings are regular 
in codimension one, so again we only have points as singularities. 
% Definiton von arithmetischen Fl?en in der Einfhrung???
We can also apply the same process to an arithmetic surface $\pi:X\,\rightarrow\,S$, 
at which $S$ is a Dedekindscheme of characteristic zero.
% Bemerkung zu Lipman bei arithmetischen Fl?en ber Dedekindschemata der Charakteristik > 0.
% normale A.F. von endl. Typ ber S hat dann ev. unendl. viele Singularit?n, da
% S nicht unbedingt excellent ist. Sing(X) ist dann insbes. nicht unbedingt abgeschl.
% Diesen Fehler kann man durch Annahme beheben, dass generische Faser glatt ist.
% Beweis in Liu06,Cor.8.3.51
Arithmetic surfaces are models of algebraic curves over number fields. 
$X$ is excellent and then Sing$(X)$ is a closed subset of $X$. 
Normalizations and blow ups of excellent schemes 
are locally of finite type and therefore excellent as well. Hence the sequence
$$ \cdots X_{i+1}\rightarrow X_i^\prime\rightarrow X_i\rightarrow\cdots\rightarrow X_1\rightarrow X_0=X$$
with normalizations $X_i^\prime\rightarrow X_i$ and blow ups $X_{i+1}\rightarrow X_i^\prime$ 
along Sing$(X_i^\prime)$ is defined.


\begin{thm}[Lip78]
Let $X$ be an excellent, Noetherian and reduced 2-dimensional scheme. Then the above
sequence is finite and $X$ has a strong desingularization.
\end{thm}

\noindent
The proof of this theorem is complex and would go beyond the scope of this article.
But we want to refer to the proof in [CS86,Ch.XI], 
% In Literaturverzeichnis von R. Klinzmann nachsehen!
which is more geometric than the original proof of Lipman.\\
An arithmetic surface $\pi:X\,\rightarrow\, S$ is of finite type over $S$ since 
the structure morphism $\pi$ is projective. Consequently $X$ is an excellent, 
Noetherian and reduced scheme of dimension $2$.
The map $\pi$ is a birational morphism of integral schemes and it particularly induces 
an isomorphism on open subsets. Therefore the generic fibres are isomorphic.\\


\begin{cor}
Let $\pi:X\,\rightarrow\, S$ be an arithmetic surface over a Dedekindscheme $S$ 
of characteristic zero. Then there exists a strong desingularization 
$X^\prime\,\rightarrow\,X$ of $X$. In particular $X^\prime$ is irreducible and 
reduced and the generic fibres $X_\eta^\prime$ and $X_\eta$ are isomorphic.
\end{cor}

\noindent
So we can construct an algorithm to desingularize excellent Noetherian surfaces 
over fields and over the integers. There are three things we need for the construction 
in particular: The computation of a blow up, the singular locus and the 
normalization of a reduced ring. Blow ups have been studied and implemented in 
{\sc{Singular}} by Anne Fr\"uhbis-Kr\"uger in the library {\sf{resolve.lib}}. 
The computation of the singular locus over the integers asks for a different 
approach than we know from computing the singular locus over a field. The 
tightest bottleneck of the algorithm though are the normalizations. 
There are three various algorithms implemented for reduced rings over fields in 
the library {\sf{normal.lib}}, {\sf{modnormal.lib}} and {\sf{locnormal.lib}} 
in {\sc{Singular}} by B\"ohm, Decker, Greuel, Pfister et al.
Two of them are implemented by us for reduced rings over the integers in the 
library {\sf{normalint.lib}}. 
We will disclose the basic concepts of the necessary computations in the 
following sections.\\

\subsection{Singular Locus}
As we are working with affine varieties and schemes we will define the singular 
locus as follows:

\begin{df}[Liu06,Def. 4.2.8]
Let $X$ be a scheme, $x \in X$ and let $\mathfrak{m}$ be the maximal ideal of the 
local ring $\mathcal{O}_{X,x}$ and $k:=\mathcal{O}_{X,x}/ \mathfrak{m}= \kappa(x)$.
Then $X$ is called regular in $x$, if and only if 
$dim_k( \mathfrak{m}/ \mathfrak{m}^2)=dim_k( \mathcal{O}_{X,x})$. The scheme $X$ is 
called regular if it is regular in all its points. We call a point $x \in X$ 
singular if $X$ is not regular in $x$.
\end{df}

\noindent
For a Noetherian ring $R$ this means algebraically spoken, that $R$ is called 
regular if all its localizations $R_P$, $P\in$ spec$(R)$ are regular.
This definition alone is not enough for the actual computation of the singular 
locus of a scheme. If we have an ideal $I$ of height $h$ over a field $k$,
the well known jacobian-criterion tells us that its singular locus  
is defined by the $h$-minors  of its jacobian ideal $J_I$ and the ideal $I$ 
itself, so Sing$(I)=$V$(\langle J_I,I \rangle)$.
If $I$ is an ideal over the integers it is not as easy to compute the singular 
locus. But we know, that if we take the radical of the above defined ideal 
$\langle J_I,I \rangle$ we get a so called test ideal $J$ of $I$. 
It can be shown that Sing$(I)\subseteq$ V$(J)$. So there are only finitely many 
points which we have to check for regularity.
% bei R.Klinzmann nach Hinweisen gucken
(We will go into other useful properties of the ideal $J$ in the section about 
the normalizations.)\\
The following lemma states a criterium to decide if a point of a scheme is regular
or singular.

\begin{lem}[Liu06, Cor. 4.2.12]
Let $(A,\mathfrak{m})$ be a regular, local ring and $f \in \mathfrak{m} \setminus \{ 0\}$.
Then $A/\langle f \rangle$ is regular if and only if $f\not\in \mathfrak{m}^2$.
\end{lem}

\begin{proof}
A is an integral ring, so $f$ is a non-zero-divisor on $A$ 
and therefore dim$(A/\langle f \rangle)=$ dim$(A)-1$. 
Let us set $\mathfrak{n}:= \mathfrak{m}/\langle f \rangle$ and $k=A / \mathfrak{m}$. Then 
we know that $(A / \langle f \rangle)$ is regular if and only if 
$dim_k( \mathfrak{n}/ \mathfrak{n}^2) = dim_k( \mathfrak{m}/ \mathfrak{m}^2)-1$. 
Because of $\mathfrak{n}/ \mathfrak{n}^2 = \mathfrak{m}/ (\mathfrak{m}^2 + \langle f \rangle)$
it follows that this is equivalent to $f\not\in \mathfrak{m}^2$.
\end{proof}

\noindent
We are working with schemes $X$ of the form $X =$ spec$(R/I)$ where $R$ is a 
Noetherian ring and $I \subset R$ is a radical ideal. 
So we can write $I = P_1 \cap\cdots\cap P_n$ with its minimal associated 
primes $P_i=\langle f_{i1},\cdots,f_{ir_i} \rangle$. (The necessary procedures
are implemented in the librabry {\sf{primdecint.lib}} in {\sc{Singular}}.)
Then spec$(R/I)=$ spec$(R/P_1)\cup\cdots\cup$ spec$(R/P_n)$ 
where spec$(R/P_i):= X_i$. 
So we can compute the singular locus of each component $X_i$ of $X$ as follows:\\ 
First we compute the corresponding test ideal $J_i$ which we decompose into its 
minimal associated primes and then we can apply the above lemma. 
For if we have a localization $R_{P_i}=\{ \frac{f}{g}\mid f\in R,\, g\not\in P_i\}$, 
then $P_i$ is the only maximal ideal of $R_{P_i}$.\\

\begin{algo}(slocusZ(I))\\

{\sc{Input}}: An ideal $I=\langle f_1,\ldots,f_k\rangle \subset
\mathbb{Z}[x_1,\ldots,x_n]$.\\
{\sc{Output}}: A list of: [1], the ideal describing the singular locus of
spec$(\mathbb{Z}[x_1,\ldots,x_n] / I)$ and [2], the components of [1]
 
% Ich habe diese Outputform gew?t, weil man fr die Normalisierung via  
% Parallelisierung sowieso den singul?n Ort zerlegen muss. Und da man hier im 
% Algorithmus erst die Komponenten berechnet und dann erst miteinander
% schneidet muss man dann ja nicht doppelt Berechnungen anstellen.
\begin{enumerate}
\item[$\bullet$] Compute the minimal associated primes $I = P_1 \cap\cdots\cap P_n$.
\item[$\bullet$] For each $i=1,\ldots,n$ compute the testideal $J_i$ of $P_i$.
\item[$\bullet$] Compute the minimal associated primes of each test ideal
$J_i = p_{i1} \cap\cdots\cap p_{is_i}$.
\item[$\bullet$] If $f_{ij} \in p_{is}^2$ for $j=1,\ldots,r_i$ and 
$s=1,\ldots,s_i$, then $p_{is}$ is singular in $\mathbb{Z}[x_1,\ldots,x_n]/I$, 
else, $p_{is}$ is regular.
\item[$\bullet$] Intersect all singular prime ideals $p_{is}$.
\end{enumerate} 
\end{algo}
 


\subsection{Blow Ups}


\subsection{Normalizations}
The normalization $\overline{A}$ of a reduced Noetherian ring $A$ is defined 
as its integral closure in the total ring of fractions $Q(A)$, which is the 
localization of $A$ with respect to the non-zerodivisors on $A$. 
$A$ is called normal if $A = \overline{A}$. One basic tool we need for the three 
normalization algorithms we present here is the notion of a test ideal. 
For this we recall the Grauert and Remmert criterion of normality:

\begin{prop}
Let A be a Noetherian reduced ring and $J\subset A$ an ideal satisfying the
following conditions:
\begin{enumerate}
\item[(1)] $J$ contains a non-zerodivisor on $A$,
\item[(2)] $J$ is a radical ideal,
\item[(3)] $N(A)\subset V(J)$, where
$$N(A) = \{P \subset A,\,\text{prime ideal} \mid A_P \,\text{is not normal}\}$$
is the non-normal locus of $A$.
\end{enumerate}
Then $A$ is normal if and only if $A \cong \text{Hom}_A(J,J)$, via the canonical map which 
maps $a \in A$ to the multiplication by $a$.
\end{prop}

\noindent
A pair $(J,x)$ with  a test ideal $J$ and $x$ a non-zerodivisor on $A$ is 
called test pair.
Such a test pair exists if and only if $\overline{A}$ is module-finite over $A$. 
We can choose any radical ideal $J$ such that $N(A)\subset V(J)$ and any 
non-zerodivisor $x\in J$, which means any element $x\in J$ if $A=R/I$ with a prime 
ideal $I$. That is one reason why we usually do a primary decomposition of a radical 
ideal $I$ and then we apply the normalization algorithms to each associated prime 
of $I$. (Another reason is that we need the pro\-perty of module-finiteness of 
the ring extensions in section 2.3.1, such that the chain of rings  becomes 
stationary.)
One option for the choice of $J$ is the radical ideal $J$ we described 
in section $2$.\\
Now we will disclose the distinctive features of the normalization 
algorithms we implemented for reduced rings over the integers.

\subsubsection{Grauert-Remmert-de Jong: Computation through an increasing chain of rings} 
{The theoretical background of this algorithm comes from the inclusions
$$A\subseteq\,\text{Hom}_A(J,J)\cong \frac{1}{x}(xJ,J)\subseteq\overline{A}\subseteq Q(A)$$
% Greuel-Pfister, Lemma 3.6.1
\noindent
where $A$ is a reduced Neotherian ring and $J\subset A$ an ideal containing a 
nonzero-divisor $x$.
The first inclusion is defined by the map $a\mapsto m_a$.
For the other inclusions we have the following Lemma:

\begin{lem} 
Let $A$ be a reduced (not necessarily Noetherian) ring, $Q(A)$ its total ring of
fractions, and $I,J$ two $A$-submodules of $Q(A)$. Assume that $I$ contains a 
non-zerodivisor $x$ on $A$.
\begin{enumerate}
\item[(1)] The map 
$$\phi:\text{Hom}_A(I,J)\rightarrow\frac{1}{x}(xJ:_{Q(A)}I)=(J :_{Q(A)}I),\,\,
\varphi\mapsto\frac{\varphi(x)}{x},$$
is independent of the choice of $x$ and an isomorphism of $A$-modules.
\item[(2)] If $J\in A$ then $(xJ:_{Q(A)}I) = (xJ:_AI)$.
\end{enumerate}
\end{lem}
\begin{proof}
\begin{enumerate}
\item[(1)] 
Let $q \in I$ be another non-zerodivisor on $A$. Write $x=x_1/x_0$ and 
$q=q_1/q_0$, with $x_0,q_0\in A$ non-zerodivisors and $x_1,q_1\in A$.
Then $c:= x_0q_0\in A$ is a non-zerodivisor and $cx,cq \in A$ with $cxq\in I$. 
Since $\varphi\in\text{Hom}_A(I,J)$ is $A$-linear, we can write
$$cx\varphi(q) = \varphi(cxq) = cq\varphi(x),$$
\noindent
so $\varphi(x)/x=\varphi(q)/q$ in $Q(A)$, showing that $\phi$ is 
independent of $x$. Moreover, for any $f \in I$ we have 
$$\frac{\varphi(x)}{x}f=\frac{\varphi(cx)}{cx}f=\frac{\varphi(cxf)}{cx}
=\frac{cx\varphi(f)}{cx}=\varphi(f)\in J$$
\noindent
in particular $\varphi(x)f \in xJ$. This shows that the image $\phi(\varphi)$ 
is in $1/x\,(xJ:_{Q(A)}I)$ and 
$\varphi(x)=0\Leftrightarrow \forall\,f\in I \,\,\,\varphi(f)=0\Leftrightarrow\varphi=0$, 
therefore $\phi$ is injective.\\
To see that $\phi$ is surjective, let $q \in Q(A)$ satisfy $qI \subset J$. 
Denote by $m_q \in \text{Hom}_A(I,J)$ the multiplication by $q$. 
Then $\phi(m_q)=qx/x=q$ showing that $\phi$ is surjective.
\item[(2)]
During the proof of (1) we have seen that 
$$(xJ :_{Q(A)}I)=\{\varphi(x) \mid \varphi\in\text{Hom}_A(I,J)\}.$$
Hence, the claimed equality holds if and only if $\varphi(x) \in A$ for all 
$\varphi\in\text{Hom}_A(I,J)$, which is clearly true if $J \subset A$.
\end{enumerate}
\end{proof}


\noindent
The inclusion $1/x\,(xJ,J)\subseteq\overline{A}$ follows from the fact 
that any $b\in(xJ:J)$ satisfies an integral relation 
$b^p+a_{p-1}b^{p-1}+\cdots+a_0=0,\, a_i\in\langle x^i\rangle$, hence $b/x$
is integral over $A$.\\
% Proposition 3.2.5 Greuel Pfister

\noindent
If $A$ is not normal, we get a proper ring extension 
$A\varsubsetneq (\text{Hom}_A(J,J) =: A_1$.
Then we check the normality of $A_1$  by applying Proposition $6$. 
If $A_1$ is not normal we obtain a new ring $A_2$ by that same proposition, 
which then has to be tested for normality, and so on. That means, we get a chain 
of inclusions of rings
$$A \subset A_1 \subset A_2 \subset \cdots$$
(with rings $A_i = A[t_1,\ldots,t_{s_i}]/I_i$, where $I_i$  is an ideal in $A_i$, 
and natural maps $\psi_i : A \hookrightarrow A_i$). 
If we get a normal ring $A_N$, then $A_N\cong \overline{A}$, because 
$\overline{A}$ is a finitely generated $A$-module.
%GLS-normalization paper Lemma 2.6
The fact which makes this algorithm practicable, is the isomorphism
$\text{Hom}_A(J,J)\cong \frac{1}{x}(xJ:_AJ)$ allowing us to compute 
$\text{Hom}_A(J,J)$ over $A$.

\begin{lem}
Let $A$ be a reduced Noetherian ring, and $(J,x)$ a test pair for $A$. Let
$\{u_0 = x, u_1,\ldots,u_s\}$ be a system of generators for the $A$-module 
$(xJ:_AJ)$. If $t_1 ,\ldots,t_s$ denote new variables, then 
$t_j \mapsto u_j/x,\,1 < j < s$, defines an isomorphism of $A$-algebras
$$A_1 := A[t_1,\ldots,t_s]/I_1\longrightarrow \frac{1}{x}(xJ :_AJ),$$
with $I_1$ the kernel of the map $t_j\mapsto u_j/x$ from 
$A[t_1,\ldots,t_s]\rightarrow\frac{1}{x}(xJ:_AJ)$.
\end{lem}
See Greuel and Pfister (2008, Lemma 3.6.7) for the computational aspects of this
lemma.\\

\noindent
If we are computing in a ring over the integers, we are talking about modules 
instead of algebras. But the algorithm works there as well, if we replace the 
procedures in the computation through integer-compatible ones.
The computation of the test ideal $J$ and the increasing number of variables
in the computation of $\text{Hom}_A(J,J)$ can become sooner unpractical than 
over a field, coefficient explosions may even interrupt the computations.
This algorithm is implemented in the library {\sf{normalint.lib}} in {\sc{Singular}} 
for reduced rings over the integers.\\}

\subsubsection{Greuel-Laplagne-Seelisch: Computation through an increasing chain of ideals}
{This algorithm [Greuel et al. 2010] (GLS normalization algorithm for short),
which is also based on the Grauert and Remmert criterion, is an improvement of
de Jongs algorithm. Most of its computations are over the original ring, so
we can avoid many computations with additional variables.
The idea is to compute ideals $U_1,\ldots,U_N\,\subset A$ and non-zerodivisors 
$d_1,\ldots,d_N$ on $A$, such that

$$A\subset\frac{1}{d_1}U_1\subset\cdots\subset\frac{1}{d_N}U_N=\overline{A}\subset Q(A)$$

\noindent
The GLS normalization algorithm is general more effective than de Jong's algorithm. 
The only computation in $\text{Hom}_A(J,J)$ is the radical of the test ideal
(GLS paper prop. 3.2).
When the characteristic of the base field is $q > 0$, it is possible to
compute also this radical over the original ring. For this, the Frobenius map is used, as
described in Matsumoto (2001).
From the construction we know that ${\frac{1}{d_i}} U_i$ is a finitely generated 
$R$-module and hence there is a surjection
$$R_i := R[t_1,t_2,\ldots,t_{s_i}]\rightarrow {\frac{1}{d_i}} U_i,\, t_j \mapsto u_j,$$
where $\{d_i, u_1,\ldots,u_s\}$ is a set of $R$-module generators of $U_i$. 
If $I_i$ denotes the kernel of this map, we get a ring isomorphism
$$\varphi_i : A_i := R_i/I_i \rightarrow {\frac{1}{d_i}} U_i \subset Q(A).$$
(Note that the definition of $I_i$ is now slightly different from the one given in 3.2.1).
To obtain $A$-module generators of $\varphi_i(J_i)$ it is not enough to
compute the images of the generators of $J_i$. But there is a simple algorithm 
to get these generators (Algorithm 2, GLS paper).\\
Once we have computed a test pair $(J_i,x_i)$ in $A_i$, the next step is to compute 
the quotient $(pJ_i :_{A_i} Ji)$. The following theorem shows that this 
computation can be carried out in the original ring $A$.

\begin{thm}
Let $A = R/I$, $A^\prime = A[t_1,\ldots,t_s]/I^\prime$  be a finite ring extension 
and maps $\psi: A \hookrightarrow A^\prime$, $\varphi : A^\prime \hookrightarrow Q(A)$. 
Let $(J,x)$ be a test pair for $A$ and $(J^\prime,x^\prime)$ a test pair for 
$A^\prime$, with $x^\prime =\psi(x)$. Let $U,H$ be ideals of $A$ and $d\in A$ such that 
$\varphi(A^\prime)=\frac{1}{d}U$ and $\varphi(J^\prime)=\frac{1}{d}H$. Then
$$(x^\prime J^\prime:_{A^\prime} J^\prime) =\frac{1}{d}(dxH :_A H).$$
\end{thm}
\begin{proof}
The proof is an easy consequence of Lemma 7. Omitting $\psi$ and $\varphi$ in 
the following notations and applying Lemma 7 to $x\in J \subset A$ we get
$$(xJ^\prime :_A J^\prime) = (p^\prime J^\prime :_{Q(A)} J^\prime) = (pH :_{Q(A)} H),$$
since $Q(A^\prime) = Q(A)$ and $J^\prime = \frac{1}{d}H$.
\noindent
On the other hand, we can apply Lemma 7 to $dp\in H\subset A$ and get
$$\frac{1}{d}(dpH :_A H) =\frac{1}{d}(dpH :_{Q(A)}H) = (pH :_{Q(A)}H).$$
\end{proof}

\noindent
So now we know how to compute the ideals $U_i\subset A$.
This algorithm works whenever Gr\"obner bases, radicals and ideal quotients can be computed in 
rings of the form $R[t_1,\cdots,t_s]$, so the implementation for reduced rings 
over the integers requested only the customary changes like taking care not to 
divide trough any coefficients and replacing particular procedures through their
relative integer-suitable ones. 
This algorithm is also implemented in the library 
{\sf{normalint.lib}} in {\sc{Singular}} for reduced rings over the integers.\\}

\noindent
In 2011 B\"ohm-Decker-Pfister-Laplagne-Steenpass-Steidel implemented a third 
algorithm for the normalization of a reduced Noetherian ring $A$ over a field. 
In view of modern multi-core computers it uses parallelization of the  GLS 
normalization via stra\-tifying Sing$(A)$.\\
If Sing$(A)= \{P_1,\ldots,P_s\}$ is finite, we set 
$S_i = A\setminus P_i$ for $i=1,\ldots,s$. Then for any intermediate ring 
$A\subset A^{(i)} \subset \overline{A}$ with $S_i^{-1}A^{(i)}=\overline{S_i^{-1}A}$ 
the equation $\overline{A}= \sum_{i=1}^s A^{(i)}$ holds.
So in the case of a reduced ring $A$ over the integers we can compute Sing$(A)$ and 
normalize locally at each $P_i$ to get $\overline{A}$.\\
All corresponding implementations for reduced rings over fields are available in 
the {\sc{Singular}} libraries {\sf{modnormal.lib}} and {\sf{locnormal.lib}}.
This computation also uses modular me\-thods and it is in general faster 
than the pure GLS-normalization. It is planned to be implemented for polynomial 
rings over the integers.


\section{Example}
We want to desingularize the surface $x^2+y^3z^3+y^2z^5\subset\mathbb{Q}[x,y,z]$ 
with Lipmans algorithm. The output is a list consisting of two lists of rings. 
The first component gives the final charts, and the second component is the list 
of all charts.\\

\begin{table}[htbp]
	\begin{tabular}{ll}
		\texttt{> LIB "lipmanresolve.lib"; } &  \texttt{// load the library lipmanresolve} \\
		\texttt{> ring r=0,(x,y,z),dp;} & \texttt{// ring of char 0 with} \\
		 &  \texttt{// variables x,y and z} \\
		\texttt{> ideal i=x\^\ \hspace*{-0.1cm}2+y\^\ \hspace*{-0.1cm}3*z\^\ \hspace*{-0.1cm}3
		+y\^\ \hspace*{-0.1cm}2*z\^\ \hspace*{-0.1cm}5;} & \\ \\
		\texttt{> list l=lipmanresolve(i);} &  \texttt{// compute the desingularization} \\ \\
		\texttt{> size(l); } &   \texttt{// the output is a list} \\
		2 &   \texttt{// of 2 components } \\ \\
		\texttt{> l[1]; } &   \texttt{// list of final charts} \\
		&   \texttt{// every chart is given by} \\
		&   \texttt{// its defining ring} \\ 
		&   \texttt{// there are three final charts}\\ \\
		\texttt{[1]:}  &   \\ 
		\texttt{// characteristic : 0}         &  \\
		\texttt{// number of vars : 2}  &   \\
		\texttt{//} \hspace*{1.2cm} \texttt{block   1 : ordering dp}  &   \\
		\texttt{//} \hspace*{2.7cm}  \texttt{: names    y(2) y(3)}  &   \\
		\texttt{//}  \hspace*{1.2cm}      \texttt{block   2 : ordering C} &   \\ \\
		\texttt{// characteristic : 0}         &  \\
		\texttt{// number of vars : 3}  &   \\
		\texttt{//} \hspace*{1.2cm} \texttt{block   1 : ordering dp}  &   \\
		\texttt{//} \hspace*{2.7cm}  \texttt{: names    x(2) y(1) y(3)}  &   \\
		\texttt{//}  \hspace*{1.2cm}      \texttt{block   2 : ordering C} &   \\ \\
		\texttt{// characteristic : 0}         &  \\
		\texttt{// number of vars : 3}  &   \\
		\texttt{//} \hspace*{1.2cm} \texttt{block   1 : ordering dp}  &   \\
		\texttt{//} \hspace*{2.7cm}  \texttt{: names    x(3) y(1) y(2)}  &   \\
		\texttt{//}  \hspace*{1.2cm}      \texttt{block   2 : ordering C} &   \\ 
	\end{tabular}
\end{table}

\begin{table}[htbp]
	\begin{tabular}{ll}

		\texttt{> l[2];  } &   \texttt{// list of all charts} \\
		\texttt{[1]:}  &   \\
		\texttt{//}   \texttt{characteristic : 0}  &   \\
		\texttt{//}    \texttt{number of vars : 3}  &   \\
		\texttt{//}    \hspace*{1.2cm}   \texttt{block   1 : ordering dp}  &   \\
		\texttt{//}      \hspace*{2.7cm}   \texttt{: names    x y z}  &   \\
		\texttt{//}      \hspace*{1.2cm}    \texttt{block   2 : ordering C}  &   \\
		\texttt{[2]:}  &   \\
		\texttt{//}    \texttt{characteristic : 0}  &   \\
		\texttt{//}    \texttt{number of vars : 3}  &   \\
		\texttt{//}     \hspace*{1cm}    \texttt{ block   1 : ordering dp}  &   \\
		\texttt{//}      \hspace*{2.7cm} \texttt{: names    T(1) T(2) T(3)}  &   \\
		\texttt{//}    \hspace*{1.2cm}      \texttt{block   2 : ordering C}  &   \\
		\texttt{// characteristic : 0}         &  \\
		\texttt{// number of vars : 2}  &   \\
		\texttt{//} \hspace*{1.2cm} \texttt{block   1 : ordering dp}  &   \\
		\texttt{//} \hspace*{2.7cm}  \texttt{: names    y(2) y(3)}  &   \\
		\texttt{//}  \hspace*{1.2cm}      \texttt{block   2 : ordering C} &   \\ \\
		\texttt{// characteristic : 0}         &  \\
		\texttt{// number of vars : 3}  &   \\
		\texttt{//} \hspace*{1.2cm} \texttt{block   1 : ordering dp}  &   \\
		\texttt{//} \hspace*{2.7cm}  \texttt{: names    x(2) y(1) y(3)}  &   \\
		\texttt{//}  \hspace*{1.2cm}      \texttt{block   2 : ordering C} &   \\ \\
		\texttt{// characteristic : 0}         &  \\
		\texttt{// number of vars : 3}  &   \\
		\texttt{//} \hspace*{1.2cm} \texttt{block   1 : ordering dp}  &   \\
		\texttt{//} \hspace*{2.7cm}  \texttt{: names    x(3) y(1) y(2)}  &   \\
		\texttt{//}  \hspace*{1.2cm}      \texttt{block   2 : ordering C} &   \\ \\
	\end{tabular}
\end{table}



\begin{table}[htbp]
To see the information which is inside a concrete ring or chart it is enough to set the selected ring and proceed as follows: \\ \\
\begin{tabular}{ll}
\texttt{> def R22=l[2][2]; } &     \texttt{ // define a new ring} \\
\texttt{> setring R22;}  &  \texttt{ // set this ring}  \\
\texttt{==== intermediate scheme:}  &   \\
\texttt{> i;}  &  \texttt{ // ideal is called i in an intermediate ring}  \\
\texttt{$_{-}$[1]=T(3)\^\ \hspace*{-0.1cm}3+T(1)\^\ \hspace*{-0.1cm}2+T(2)*T(3)}  
&   \texttt{ //  R22/i is the surface after a normalization}  \\
\texttt{==== Images of variables of original ring:}  & \texttt{ // morphism defining}  \\
\texttt{> liph;}  &   \texttt{ // the normalization}  \\
\texttt{$_{-}$[1]=-T(1)*T(2)*T(3)} &   \\
\texttt{$_{-}$[2]= T(2)} &   \\
\texttt{$_{-}$[3]= T(3)} &   \\
\texttt{> print(Path);}  &   \texttt{ //  path to this chart}   \\
\texttt{ 0,1,}  &   \\
\texttt{ 0,1,}  &   \\
\texttt{ 1,1,}  &   \\
\end{tabular}
\texttt{  
	// from [0,0,1] ([initial ring, initial vertex,first branch])\\
	// to [1,1,1] ([normalization, vertex, branch of vertex above]) 
	}
\end{table} 

\begin{table}[htbp]
	\begin{tabular}{ll}
		\texttt{> def R13=l[1][3]; } &     \texttt{ // define a new ring} \\
		\texttt{> setring R13;}  &  \texttt{ // set this ring}  \\
		\texttt{==== First final chart:}  &   \\
		\texttt{> lip;}  &  \texttt{ // }  \\
		\texttt{$_{-}$[1]=x(3)*y(1)\^\ \hspace*{-0.1cm}2+y(1)*y(2)+1}  & \texttt{ // morphism defining}   \\
		\texttt{==== Images of variables of original ring:}  & \texttt{ // morphism defining}  \\
		\texttt{> liph;}  &  \texttt{ // the normalization and the blow up }  \\
		\texttt{$_{-}$[1]=-x(3)\^\ \hspace*{-0.1cm}5*y(1)\^\ \hspace*{-0.1cm}2*y(2)}  &   \\
		\texttt{
		-2*x(3)\^\ \hspace*{-0.1cm}4*y(1)*y(2)\^\ \hspace*{-0.1cm}2
		-x(3)\^\ \hspace*{-0.1cm}3*y(2)\^\ \hspace*{-0.1cm}3}  &   \\
		\texttt{$_{-}$[2]=-x(3)\^\ \hspace*{-0.1cm}2*y(1)*y(2)
		-x(3)*y(2)\^\ \hspace*{-0.1cm}2}  &   \\
		\texttt{$_{-}$[2]= x(3)}  &   \\ 
		\texttt{==== The exceptional divisor:}  &   \\
		\texttt{> eD;}  &   \texttt{ // the blow up}  \\
		\texttt{$_{-}$[1]=x(3)} &   \\
		\texttt{$_{-}$[2]=y(1)*y(2)+1} &   \\
		\texttt{> print(Path);}  &   \texttt{ // path to this chart}   \\
		\texttt{ 0,1,-1}  &   \\
		\texttt{ 0,1,4}  &   \\
		\texttt{ 1,1,3}  &   \\
	\end{tabular}
	\texttt{ 
		// from [0,0,1] ([initial ring, initial vertex,first branch])  \\
		// over [1,1,1] ([normalization, vertex, branch of vertex above]] \\
		// to [-1,4,3] ([blow up, vertex, branch of vertex above]) 
		}
\end{table} 



\begin{thebibliography}{4}

\bibitem{GLS paper}
G.-M. Greuel, S. Laplagne, F. Seelisch:
 {\em Normalization of Rings}. Preprint submitted to Elsevier 13 January 2010

\bibitem{Parallel Normalization}
J. B\"ohm, W. Decker, G. Pfister, S. Laplagne, A. Steenpass, S. Steidel:
 {\em Parallel Algorithms for Normalization}. arXiv:1110.4299v1 [math.AC] 19 Oct 2011

\bibitem{GP}
G.-M. Greuel, G. Pfister:
 {\em A {\sc{singular}} Introduction to Commutative Algebra}.  
 In {\it ICMS 2008}, Second Edition, pp $244-258$, 2008. Springer-Verlag Berlin Heidelberg, 2008.
 
\bibitem{Matsumoto} 
Matsumoto, R., 2001. {\em Computing the radical of an ideal in positive characteristic.} 
J. Symbolic Comput. 32 (3), 263-271.

\end{thebibliography}


\end{document}
